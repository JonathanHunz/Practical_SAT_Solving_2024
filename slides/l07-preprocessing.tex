\documentclass[t]{sdqbeamer}
%\documentclass[c]{sdqbeamer} 

\usepackage{listings}
\usepackage{graphicx}
\usepackage{tabularx}
\usepackage{multirow}
\usepackage{multicol}
\usepackage{tabulary}
\usepackage{colortbl}
\usepackage{tikzsymbols}
\usepackage{tikz}
\usetikzlibrary{positioning,fit,shapes}
\usepackage[lined,linesnumbered,ruled,noend]{algorithm2e}
\usepackage{bm}
\usepackage{enumitem}
\setlist[enumerate]{label*=\bf\alph*),ref=\alph*}
\setlist[itemize]{label=\textbullet}

\hypersetup{
	colorlinks=true,
	urlcolor=kit-orange
}

% set sdqbeamer options
\titleimage{blender-render}
\groupname{Algorithm Engineering}
\grouplogo{ae}
\selectlanguage{english}

% define title etc.pp.
\title[SAT Solving]{Practical SAT Solving}
\subtitle{Lecture 6}
\author{\underline{Markus Iser}, Dominik Schreiber, Tom\'a\v{s} Balyo}
\date{May 27, 2024}

% Existing KIT colors: kit-green, kit-blue, kit-red, kit-gray, kit-orange, kit-lightgreen, kit-brown, kit-purple, kit-cyan
% configure appearance
\setbeamercolor{block title}{bg=kit-blue}
\setbeamercolor{block body}{bg=kit-blue!10}
\setbeamercolor{block title example}{bg=kit-orange}
\setbeamercolor{block body example}{bg=kit-orange!10}
\setbeamertemplate{itemize item}{\color{kit-gray}\textbullet}
\setbeamertemplate{itemize subitem}{\color{kit-gray}\textbullet}
\setbeamercolor{item projected}{bg=kit-gray, fg=kit-gray}
\renewcommand{\insertnavigation}[1]{} % remove navigation bar

% define commands
\definecolor{myblue}{HTML}{0D3B66}
\definecolor{myred}{HTML}{6E0E0A}
\definecolor{mypink}{HTML}{F7B2B7}

\newcommand{\vars}[1]{\textsf{vars} (#1)}
\newcommand{\lits}[1]{\textsf{lits} (#1)}
\newcommand{\clss}[1]{\textsf{clss} (#1)}

\newcommand{\highl}[1]{\textcolor{myblue}{#1}}
\newcommand{\highlo}[1]{\textcolor{myred}{#1}}
\newcommand{\highlow}[1]{\textcolor{mypink}{#1}}

% Extra column types for tabularx
\newcolumntype{C}{>{\centering\arraybackslash}X}
\newcolumntype{L}{>{\raggedright\arraybackslash}X}
\newcolumntype{R}{>{\raggedleft\arraybackslash}X}

\newcommand{\setcolsep}[1]{\setlength{\tabcolsep}{#1}}
\newcommand{\setrowsep}[1]{\renewcommand{\arraystretch}{#1}}

% Definitions for the Tseitin transformation
\newcommand{\true}{\ensuremath{\mathit{True}}}
\newcommand{\false}{\ensuremath{\mathit{False}}}
\newcommand{\allvars}{\ensuremath{\mathcal{V}}}
\newcommand{\tseitin}[1]{\ensuremath{\mathcal{T}(#1)}}
\newcommand{\tseitinRec}[2]{\ensuremath{\mathcal{T}^{#2}(#1)}}
\newcommand{\tseitinSym}[1]{\ensuremath{\mathcal{T}_\mathsf{lit}(#1)}}
\newcommand{\tseitinDef}[2]{\ensuremath{\mathcal{T}_\mathsf{def}^{#2}(#1)}}
\newcommand{\hcancel}[2][black]{\setbox0=\hbox{$#2$}\rlap{\raisebox{.45\ht0}{\textcolor{#1}{\rule{\wd0}{1pt}}}}#2} 
\newcommand{\sateq}{\mathrel{\overset{\makebox[0pt]{\mbox{\normalfont\tiny\sffamily SAT}}}{=}}}

\newcommand{\enc}{\ensuremath{\mathcal{E}}} % encoding

% exercise commands
\newcommand{\exhead}[3]{
\hrule~\\[1ex]\noindent
{\bf Practical SAT Solving} (ST 2024) \hfill \fbox{Assignment #1} \\[1ex]
Markus Iser, Dominik Schreiber, Tom\'a\v{s} Balyo\\[1ex]
Algorithm Engineering (KIT) \hfill #2 -- #3\\
\hrule
\thispagestyle{empty}
}


\begin{document}
\begin{frame}
	\thispagestyle{empty}
	\titlepage
\end{frame}


\begin{frame}{Recap}
    \begin{block}{Lecture 6: Modern SAT Solving 2}
		\begin{itemize}\setlength{\itemsep}{1ex}
			\item Efficient Unit Propagation
			\item Clause Forgetting
			\item Modern Decision Heuristics: VSIDS \& Co.
		\end{itemize}
    \end{block}
    % \pause
	\begin{block}{Today}
        Preprocessing
	\end{block}
\end{frame}


\begin{frame}{Conflict-driven Clause Learning (CDCL) Algorithm}
\vspace*{-1em}
\begin{columns}[T]
\begin{column}{.3\linewidth}
~\\[1em]
\highl{Last Time}
\begin{itemize}%\setlength{\itemsep}{1em}
    \item Efficient Unit Propagation
    \item Clause Forgetting
    \item Modern Decision Heuristics
\end{itemize}
\highlo{Today}
\begin{itemize}
    \item Preprocessing
\end{itemize}
\end{column}
\begin{column}{.6\linewidth}
\begin{algorithm}[H]
    \DontPrintSemicolon
    \caption{CDCL(CNF Formula $F$, \&Assignment A $\leftarrow \emptyset$)}
    
    \SetKwFunction{propagation}{\highl{UNIT PROPAGATION}}
    \SetKwFunction{branching}{\highl{BRANCHING}}
    \SetKwFunction{conflictanalysis}{CONFLICT-ANALYSIS}
    \SetKwFunction{restart}{RESTART}
    \SetKwFunction{cleanup}{\highl{CLEANUP}}
    \SetKwFunction{preprocessing}{\highlo{PREPROCESSING}}

    \SetKwData{UNSAT}{UNSAT}
    \SetKwData{SAT}{SAT}
    
    \lIf {not \preprocessing} {
        \Return \UNSAT
    }
    \While {A is not complete} {
        \propagation\;
        \If {A falsifies a clause in $F$} {
            \lIf {decision level is $0$} {
                \Return \UNSAT
            }
            \Else {
                (clause, level) $\leftarrow$ \conflictanalysis\;
                add clause to $F$ and backtrack to level\;
                \textbf{continue}\;
            }
        }
        \lIf {\restart} {
            backtrack to level $0$
        }
        \lIf {\cleanup} {
            forget some learned clauses
        }
        \branching\;
    }
    \Return \SAT\;
\end{algorithm}
\end{column}
\end{columns}
\end{frame}


\begin{frame}{Preprocessing}
Preprocessing takes place between problem encoding and its solution.\\[1ex]
\begin{block}{Preprocessing is \dots}
\begin{itemize}
    \item a form of \highlo{reencoding} a problem: to fix bad encodings
    \item a form of \highlo{reasoning} itself: inprocessing
\end{itemize}
\end{block}
\textbf{Conjecture:} Smaller problems are easier to solve 
$\Longrightarrow$ Try to reduce the size of the formula.\\[1ex]
\begin{exampleblock}{Classic Preprocessing Techniques}
\begin{itemize}\setlength{\itemsep}{1ex}
    \item Subsumption
    \item Self-subsuming Resolution
    \item (Bounded) Variable Elimination (BVE)
\end{itemize}
\end{exampleblock}
\end{frame}


\begin{frame}{Preprocessing: Subsumption}
A clause $C$ is subsumed by $D$ iff $D \subseteq C$.\\[1ex]
Subsumed clauses can be removed from the formula without changing satisfiability: $\forall D \subseteq C, D \models C$
\begin{exampleblock}{Example}
    $\{a, b\}$ subsumes $\{a, b, c\}$ and $\{a, b, d\}$
\end{exampleblock}
\pause
\begin{block}{Implementation 1: Forward Subsumption}
Select clause $C$ and \highlo{check if it is subsumed} by any other clause $D \subseteq C$.
\begin{itemize}\setlength{\itemsep}{1ex}
    \item Temporarily mark all literals in $C$ as unsatifisfied, and use \highl{one-watched literal data-structure} to find subsumed clauses
    \item \textbf{Optimization 1:} Watch literals with the fewest occurrences
    \item \textbf{Optimization 2:} Keep literals sorted and perform merge-sort style subset check
\end{itemize}
\end{block}
\end{frame}


\begin{frame}{Preprocessing: Subsumption}
\begin{block}{Implementation 2: Backward Subsumption}
\begin{columns}[T]
\begin{column}{.45\linewidth}~\\
    Select clause $D$ and \highlo{check if it subsumes} any other clause $C \supseteq D$.\\[1ex]
    \highl{Learned clauses} are never subsumed but \highl{can subsume other clauses}, e.g., recently learned clauses
    \begin{itemize}\setlength{\itemsep}{1ex}
        \item \textbf{Optimization 1:} Check the clauses of the variable with the fewest occurrences (scales to large formulas)
        \item \textbf{Optimization 2:} Use signatures to skip the majority of subsumption checks (cf. Bloom filters)
    \end{itemize}
\end{column}
\begin{column}{.5\linewidth}
    \begin{algorithm}[H]
        \DontPrintSemicolon
        \caption{Signature-based Subsumption Check}
        \tcp{Initialization:}
        \For {$clause \in formula$} {
            clause.signature = 0\;
            \For {$lit \in *clause$} {
                clause.signature |= $1\text{ull} << \bigl(\mathop{id}(lit) \% 64\bigr)$\;
            }
        }
        \BlankLine
        \tcp{Subsumption Check:}
        \If {$D.signature~ \& \mathop{invert}(C.signature) == 0$} {
            \tcp{Check if $D$ subsumes $C$}
        }
    \end{algorithm}
\end{column}
\end{columns}
\end{block}
\end{frame}
    
    
\begin{frame}{Preprocessing: Self-Subsuming Resolution}
Applicable if the resolvent of $C$ and another clause $D$ subsumes $C$.\\[1ex]
If $C \otimes_x D \subseteq C$ then $C$ can be replaced by $C \otimes_x D$.
\begin{exampleblock}{Example}
    Let $\otimes_f$ be the resolution operator on variable $f$.\\[1ex]
    $C := \{ \lnot b, \lnot e, {\color{myblue} f}, \lnot h \}$ \qquad
    $D := \{ \lnot b, \lnot e, {\color{myblue} \lnot f} \}$ \qquad
    $E := C \otimes_f D = \{ \lnot b, \lnot e, \lnot h \}$ \\[1ex]
    $\bm\longrightarrow$ Replace $C$ by $E$ (``clause strengthening'')
\end{exampleblock}
\begin{block}{Implementation}
\begin{itemize}\setlength{\itemsep}{1ex}
    \item \textbf{Integrate with subsumption:} Allow at most one literal of D to occur negated in C
    \item \textbf{Variant:} On-the-fly subsumption/strengthening of reason clauses during conflict analysis
\end{itemize}
\end{block}
\end{frame}
    
    
\begin{frame}{Preprocessing: Bounded Variable Elimination}
Let $S_x, S_{\overline x} \subseteq F$ be the sets of all clauses containing $x$ resp. ${\overline x}$, and let $R = \{ C \otimes_x D ~|~ C \in S_x, D \in S_{\overline x} \}$ be the set of all resolvents on $x$.
%
The formulas $F$ and \highl{$F' := (F \setminus (S_x \cup S_{\overline x})) \cup R$} are \highlo{equisatisfiable} but not equivalent.\\[1ex]
%
Most important preprocessing technique in practical SAT solving
\begin{block}{Bounded Variable Elimination (BVE)}
Eliminate variable only if the formula \highlo{size does not increase} (too much).
\begin{itemize}\setlength{\itemsep}{1ex}
    \item \textbf{Note 1:} Variables of removed clauses have to be rescheduled for further elimination attempts
    \item \textbf{Note 2:} Resolvent can trigger further subsumptions and vice versa
    \item Particularly effective in presence of functional definitions (cf. Tseitin encoding)
    \item \textbf{Variant:} Incrementally Relaxed BVE: Increase bound each round if formula size did not increase too much
    \item \textbf{Optimizations:} Perform check only for bounded clause size, resolvent size, or variable occurrence count
\end{itemize}
\end{block}
\end{frame}

    
\begin{frame}{Preprocessing: Blocked Clause Elimination (BCE)}
A clause $\{ x \} \cup C$ is blocked in $F$ by $x$ if either $x$ is \highlo{pure} in $F$ or\\
for every clause $\{ \lnot x \} \cup D$ in $F$ the resolvent $C \cup D$ is a \highlo{tautology}.\\[1ex]
$\rightarrow$ Dead ends in the resolution graph, no proof beyond this point.\\[1ex]
Blocked clause elimination (BCE) has a unique fixpoint, and \highlo{preserves satisfiability}.
\begin{exampleblock}{Example}
$F := (a \lor b) \land (a \lor \lnot b \lor \lnot c) \land (\lnot a \lor c)$\\[1ex]
First clause is not blocked, second is blocked by both $a$ and $\lnot c$, third is blocked by $c$.
\end{exampleblock}
\begin{itemize}\setlength{\itemsep}{1ex}
    \item Effectiveness of BVE can be increased by interleaving it with BCE
    \item Relationship with \href{https://doi.org/10.1007/s10817-011-9239-9}{circuit-level simplification techniques}
    \item \textbf{Generalization: Covered Clauses}\\
    A clause is covered if it can be \highlo{turned into a blocked clause} by adding a covered literal.
    A literal $x$ is covered by a clause $C$, if it contains a literal $y$ such that all non-tautological resolvents of $C$ on $y$ contain $x$.
\end{itemize}
\end{frame}


\begin{frame}{Preprocessing: Solution Reconstruction}
Many preprocessing techniques remove clauses or variables from a formula in a mere satisfiability-preserving way, such that the solution to the preprocessed formula might not be a solution to the original formula.\\[1ex]
\begin{block}{Reconstruction Algorithm}
\begin{columns}[T]
\begin{column}{.35\linewidth}
~\\
Keep track of eliminated variables (BVE) and clauses (BCE) in a solution reconstruction stack $S$, and if a model is found, use it to reconstruct a solution to the original formula.
\end{column}
\begin{column}{.55\linewidth}
\begin{algorithm}[H]
\caption{Solution Reconstruction}
\KwData{Assignment $A$, Stack $S$}
\While {$S$ is not empty} {
    remove the last literal-clause pair $(l,C)$ from $S$\;
    \If {$C$ is not satisfied by $A$} {
        $A := (A \setminus \{l = 0\}) \cup \{l = 1\}$
    }
}
\BlankLine
If variables remain unassigned in $A$, then assign them an arbitrary value.
\end{algorithm}
\end{column}
\end{columns}
\end{block}
\end{frame}


\begin{frame}{Recap.}
\begin{block}{Preprocessing: Classic Techniques}
\begin{itemize}\setlength{\itemsep}{1ex}
    \item Subsumption and Self-subsuming Resolution
    \item Bounded Variable Elimination
    \item Blocked Clause Elimination
    \item Solution Reconstruction
\end{itemize}
\end{block}
\begin{block}{Next Up}
Relationship between preprocessing techniques and gate encodings
\end{block}
\end{frame}


\begin{frame}{Preprocessing: Relationship with Gate Encodings}
Tseitin encoding $E$ of a gate with output $o$, function $g$, and input literals $x_1, \dots, x_n$: $$E \equiv o \leftrightarrow g(x_1, \dots, x_n)$$ 
\vspace*{-1em}
\begin{block}{Properties of Gate Encodings}
Let a Tseitin encoding $E \equiv o \leftrightarrow g(x_1, \dots, x_n)$ be given, and let $A(X) := \{ T \cup \{ \overline x \mid x \in X \setminus T \} \mid T \in 2^X \}$ denote the set of all assignments to variables in $X$.\\[1ex]
For each input assignment $I \in A(\{x_1, \dots, x_n\})$, \\[1ex]
\begin{enumerate}\setlength{\itemsep}{1ex}
\item there exists \highlo{at least one} output assignment $O \in \{o, \overline o\}$ such that $I \cup O \models E$ \quad (left-totality)
\item there exists \highlo{at most one} output assignment $O \in \{o, \overline o\}$ such that $I \cup O \models E$ \quad (right-uniqueness)\\[1ex]
\end{enumerate}
$\bm\rightarrow$ The output is uniquely determined by the input, such that either $I, o \models E$ and $I, \overline o \not\models E$ or vice versa.
\end{block}
\end{frame}


\begin{frame}{Preprocessing: Relationship with Gate Encodings}
From the left-totality it follows that a Tseitin encoding $E$ is a satisfiable set of blocked clauses.\\[1em]
\begin{block}{Left-Totality of Gate Encodings}
Let a \highlo{Tseitin encoding $E \equiv o \leftrightarrow g(x_1, \dots, x_n)$} be given, it holds that\\[1em]
\begin{enumerate}\setlength{\itemsep}{1ex}
\item for \highlo{each clause $C \in E$}, either $o \in C$ or $\overline o \in C$\\[1ex]
\textbf{Proof:} The existence of a clause $C \in E$ such that $o \not\in \vars{C}$ would contradict left-totality, because the assignment falsifiying $C$, falsifies $E$ for any assignment to $o$.
\item and \highlo{all resolvents $R \in E_o \otimes_o E_{\overline o}$} are tautological.\\[1ex]
\textbf{Proof:} The existence of a non-tautological resolvent $R \in E_o \otimes_o E_{\overline o}$ would contradict left-totality, because $E \models R$ and $o \not\in \vars{R}$, such that the assignment falsifying $R$, falsifies $E$ for any assignment to $o$.
\end{enumerate}
\end{block}
\end{frame}


\begin{frame}{Preprocessing: Relationship with Gate Encodings}
From the left-totality it follows that a Tseitin encoding $E$ is a satisfiable set of blocked clauses.\\[1em]
\begin{example}[Tseitin encoding $E \equiv o \leftrightarrow x \land y$]
Let a Tseitin encoding $E := \bigl\{\{ \lnot o, x \}, \{ \lnot o, y \}, \{ o, \lnot x, \lnot y \} \bigr\} \equiv o \leftrightarrow x \land y$ be given, it holds that\\[1em]
\begin{enumerate}\setlength{\itemsep}{1ex}
\item all resolvents in $E_o \otimes_o E_{\overline o} = \bigl\{ \{ x, \lnot x, \lnot y \}, \{ y, \lnot x, \lnot y \} \bigr\} \equiv \top$ are tautological,
\item and Blocked Clause Elimination (BCE) would remove all clauses from $E$.\\[1ex]
\end{enumerate}
\end{example}~\\[-1ex]
\textbf{Questions:}\\[1ex]
\begin{itemize}\setlength{\itemsep}{1ex}
\item What does BCE do to $F = \bigl\{ \{ o \} \bigr\} \cup E$?
\item What does BCE do to $F = \bigl\{ \{ \lnot o \} \bigr\} \cup E$?
\item What does BCE do to $F = \bigl\{ \{ q \}, \{ \lnot q, o, p \}, \{ \lnot q, \lnot o, \lnot p \} \bigr\} \cup E$?
\end{itemize}
\end{frame}


\begin{frame}{Preprocessing: Relationship with Gate Encodings}
Resolving the clauses of a gate encoding on the output literal $o$ results in a set of tautological clauses.\\[1em]
\begin{block}{Idea: Optimized Variable Elimination for Gate Encodings $E$}
Let a formula $F = E \cup R$ with gate clauses $E$ and remainder $R$ be given.\\
Apply variable elimination as follows:
\vspace*{-1ex}
\begin{align*}
( E_x \cup R_x) \otimes (E_{\overline x} \cup R_{\overline x}) 
%[1ex]
&\equiv (E_x \otimes R_{\overline x}) \cup (R_x \otimes E_{\overline x}) \cup (R_x \otimes R_{\overline x}) \cup (E_x \otimes E_{\overline x})\\[1ex]
%
&\equiv (E_x \otimes R_{\overline x}) \cup (R_x \otimes E_{\overline x}) \cup (R_x \otimes R_{\overline x}) \tag{$E_x \otimes E_{\overline x} \equiv \top$}\\[1ex]
&\equiv (E_x \otimes R_{\overline x}) \cup (R_x \otimes E_{\overline x})
\tag{$(E_x \otimes R_{\overline x}) \cup (R_x \otimes E_{\overline x}) \models R_x \otimes R_{\overline x}$}
\end{align*}
\end{block}~\\[-1ex]
\textbf{Proof Idea:} Each clause $c \in R_x \otimes R_{\overline x}$, derived by resolving $c_x \in R_x$ and $c_{\overline x} \in R_{\overline x}$, can also be derived by resolving clauses in $R_{\overline x} \otimes E_x$ and $E_{\overline x} \otimes R_x$.
\end{frame}
    

\begin{frame}{Preprocessing: Unit Propagation, Probing}
\begin{block}{Propagation-based Redundancy}
Let a formula $F$, a clause $C \in F$, and a literal $x \in C$ be given.\\[1ex]
\begin{itemize}\setlength{\itemsep}{1ex}
    \item \textbf{Failed Literal Probing}\\[1pt]
    If $F \land x \vdash_{\mathop{UP}} \bot$, then $F \models \lnot x$.\\$\implies$ add $\{ \lnot x \}$ to $F$
    \item \textbf{Asymmetric Literal Elimination (ALE)}\\[1pt]
    If $F \setminus C \land \overline{C \setminus \{ x \}} \vdash_{\mathop{UP}} \overline x$, then $F \models C \setminus \{ x \}$.
    \\$\implies$ strengthen $C$ to $C \setminus \{ x \}$
    \item \textbf{Asymmetric Tautology Elimination (ATE)}\\[1pt]
    If $F \setminus C \land \overline C \vdash_{\mathop{UP}} \bot$, then $F \models C$.
    \\$\implies$ remove $C$ from $F$
\end{itemize}
\end{block}
\end{frame}


\begin{frame}{Preprocessing: Unit Propagation, Probing}
\begin{block}{Optimizations of Probing Techniques}~\\
\begin{itemize}\setlength{\itemsep}{1em}
    \item \textbf{Restricted Form of ATE/ALE}\\[1pt]
    Hidden Tautology Elimination (HTE), Hidden Literal Elimination (HLE) only propagate over binary clauses.
    \item \textbf{Avoidance of Redundant Propagations}\\[1pt]
    Sort literals and clauses in a formula to simulate a trie, and reuse propagations that share the same prefix.
    % \item \textbf{Unhiding}\footnote{\href{https://link.springer.com/chapter/10.1007/978-3-642-21581-0_17}{2011, Heule et al., Efficient CNF Simplification Based on Binary Implication Graphs}}\\[1pt] Avoid quadratic runtime of repeated unit propagation through randomized depth-first traversal of
    the binary implication graph and application of the \emph{parenthesis theorem}.
    \item \textbf{Variants: Distillation / Vivification}\\[1pt]
    Interleave assignments and propagations to detect ATs / ALs early.
\end{itemize}~\\
\end{block}
\end{frame}


\begin{frame}{Probing: Relationship with Proof Checking}
\begin{block}{Generalizations of Blocked Clauses}~\\
\begin{itemize}\setlength{\itemsep}{1em}
    \item \textbf{Reverse Unit Propagation (RUP)}\\[1pt]
    A clause has the property RUP if and only if it is an Asymmetric Tautology (AT).
    All learned clauses are RUP at the moment of their learning.
    RUP checking is basic proof checking.
    \item \textbf{Resolution Asymetric Tautologies (RATs)}\\[1pt]
    A clause $C$ is a RAT in a formula $F$ if it contains a literal $x$ such that \highlo{each resolvent in $C \otimes_x F_{\overline x}$ is an asymmetric tautology}.
    Modern proof systems are based on RATs.
\end{itemize}
~\\
\end{block}
\end{frame}


\begin{frame}{Recap.}
\begin{block}{Recap.}
\begin{itemize}
    \item Classic Preprocessing Techniques: Subsumption, Self-subsuming Resolution, Bounded Variable Elimination, Blocked Clause Elimination
    \item Relationship between Preprocessing Techniques and Gate Encodings
    \item Probing Techniques: Failed Literal Probing, Asymmetric Literal Elimination, Asymmetric Tautology Elimination
    \item Relationship between Probing Techniques and Proof Checking
\end{itemize}
\end{block}
\begin{block}{Next Up}
Scheduling of Preprocessing Techniques, Inprocessing, Autarky Reasoning
\end{block}
\end{frame}


\begin{frame}{Preprocessing: Scheduling of Preprocessing Techniques}
At a point where one technique is unable to make further progress, another technique might be applicable and even modify the problem in a way that the first technique can make further progress.\\[1ex]

\begin{block}{Scheduling of Preprocessing Techniques}
\begin{itemize}\setlength{\itemsep}{1ex}
    \item \textbf{Heuristic Limits}\\[1pt]
    Bound the number of applications of a technique.
    \item \textbf{Scheduling of Techniques}\\[1pt]
    Non-trivial, benefit of techniques depends on the formula.
    \item \textbf{Interleaving of Techniques}\\[1pt]
    Apply techniques in a round-robin fashion.
    \item \textbf{Inprocessing}\\[1pt]
    Interleave search and preprocessing.
\end{itemize}
\end{block}
\end{frame}


\begin{frame}{Inprocessing}
\begin{block}{Idea: Interleave search and preprocessing}
\begin{itemize}\setlength{\itemsep}{1em}
\item Preprocessing can be extremely beneficial: most solvers in SAT competitions use bounded variable elimination, subsumption and self-subsuming resolution
\item Problem: Many preprocessing techniques, though polynomial, require considerable time
\item Possible Solution:
\begin{itemize}\setlength{\itemsep}{1ex}
\item Interrupt preprocessing techniques after some time
\item Resume on restart
\item Limit preprocessing time in relation to search
\end{itemize}
\item \textbf{Discussion:} What are the problems that can arise in practice when SAT instances are solved incrementally?
\end{itemize}
\end{block}
\end{frame}

    
\begin{frame}{Inprocessing: Autarkies}
Autarky reasoning is used by state-of-the-art SAT solvers (cf. \texttt{Kissat}) to remove clauses from a formula.
\begin{block}{Autarky-based Clause Removal}
Let a formula $F$ and a partial assignment $A$ be given.\\[1ex]
\begin{itemize}\setlength{\itemsep}{1ex}
    \item A clause $C \in F$ is \highlo{touched} by $A$ if it contains the negation of a literal assigned in $A$
    \item A clause $C \in F$ is \highlo{satisfied} by $A$ if it contains a literal assigned to $\true$ by $A$
\end{itemize}
An autarky is a partial assignment $A$ such that \highlo{all touched clauses are satisfied}.\\
All clauses touched by an autarky can be removed.\\[1em]
\textbf{Discussion:} How to obtain partial assignments to probe for autarky-based clause removal?
\end{block}
\begin{exampleblock}{Autarky-based Clause Removal}
The partial assignment $A = \{ \lnot a, \lnot c \}$ is an autarky for $F := \bigl\{\{ \lnot a, b \}, \{ \lnot a, c \}, \{ a, \lnot b, \lnot c \}\bigr\}$
\end{exampleblock}
\end{frame}


\begin{frame}{Techniques that do not Reduce the Formula Size}
\begin{block}{Next Time}
\begin{itemize}\setlength{\itemsep}{1em}
\item \textbf{Resolution Calculus} (a.k.a. Clause Learning)
\item \textbf{Tseitin's Extension Rule}: Introduce definitions of new variables as a conjunction of existing literals (a.k.a. Bounded Variable Addition (BVA)). Some formulas have refutations of exponential size in the resolution calculus, but of polynomial size in extended resolution, e.g., pigeonhole formulas, mutilated chessboard, \dots\\[1ex]
$\bm\rightarrow$ \texttt{SBVA-CaDiCaL}: Winner of SAT Competition 2023
\item \textbf{Symmetry Breaking Predicates}: Exclusion of Symmetric Solutions\\[1ex]
$\bm\rightarrow$ \texttt{BreakId-Kissat}: Special Price at SAT Competition 2023
\item \textbf{PReLearning}: Preprocessing adds specific Propagation Redundant (PR) clauses\\[1ex]
$\bm\rightarrow$ \texttt{KissatMAB-Prop}: Winner of SAT Competition 2023 on UNSAT instances
\end{itemize}
\end{block}
\end{frame}


% \begin{frame}{Hybrid Backtracking}
%     \begin{block}{Recent Revival of Chronological Backtracking}
%     \begin{itemize}
%     \item Nadel \& Ryvchin, SAT 2018
%     \item Backtrack chronologically iff number of untouched decision levels is higher than a given bound
%     \item SAT Competition 2018: \texttt{MapleLCMDistChronoBT} (best score)
%     \item SAT Race 2019: \texttt{MapleLCMDistChronoBT} \& \texttt{CaDiCal} (best scores)
%     \item Hard to implement: Decision levels no longer monotonically increasing on trail
%     \end{itemize}
%     \end{block}
% \end{frame}

\end{document}
