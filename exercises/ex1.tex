\documentclass{article}

\usepackage{listings}
\usepackage{graphicx}
\usepackage{tabularx}
\usepackage{tikzsymbols}
\usepackage{hyperref}
\usepackage{amsmath}

\usepackage{titlesec}
\titleformat*{\section}{\large\bfseries}

\definecolor{myblue}{HTML}{0D3B66}
\definecolor{myred}{HTML}{6E0E0A}
\definecolor{mypink}{HTML}{F7B2B7}

\newcommand{\vars}[1]{\textsf{vars} (#1)}
\newcommand{\lits}[1]{\textsf{lits} (#1)}
\newcommand{\clss}[1]{\textsf{clss} (#1)}

\newcommand{\highl}[1]{\textcolor{myblue}{#1}}
\newcommand{\highlo}[1]{\textcolor{myred}{#1}}
\newcommand{\highlow}[1]{\textcolor{mypink}{#1}}

% Extra column types for tabularx
\newcolumntype{C}{>{\centering\arraybackslash}X}
\newcolumntype{L}{>{\raggedright\arraybackslash}X}
\newcolumntype{R}{>{\raggedleft\arraybackslash}X}

\newcommand{\setcolsep}[1]{\setlength{\tabcolsep}{#1}}
\newcommand{\setrowsep}[1]{\renewcommand{\arraystretch}{#1}}

% Definitions for the Tseitin transformation
\newcommand{\true}{\ensuremath{\mathit{True}}}
\newcommand{\false}{\ensuremath{\mathit{False}}}
\newcommand{\allvars}{\ensuremath{\mathcal{V}}}
\newcommand{\tseitin}[1]{\ensuremath{\mathcal{T}(#1)}}
\newcommand{\tseitinRec}[2]{\ensuremath{\mathcal{T}^{#2}(#1)}}
\newcommand{\tseitinSym}[1]{\ensuremath{\mathcal{T}_\mathsf{lit}(#1)}}
\newcommand{\tseitinDef}[2]{\ensuremath{\mathcal{T}_\mathsf{def}^{#2}(#1)}}
\newcommand{\hcancel}[2][black]{\setbox0=\hbox{$#2$}\rlap{\raisebox{.45\ht0}{\textcolor{#1}{\rule{\wd0}{1pt}}}}#2} 
\newcommand{\sateq}{\mathrel{\overset{\makebox[0pt]{\mbox{\normalfont\tiny\sffamily SAT}}}{=}}}

\newcommand{\enc}{\ensuremath{\mathcal{E}}} % encoding

% exercise commands
\newcommand{\exhead}[3]{
\hrule~\\[1ex]\noindent
{\bf Practical SAT Solving} (ST 2024) \hfill \fbox{Assignment #1} \\[1ex]
Markus Iser, Dominik Schreiber, Tom\'a\v{s} Balyo\\[1ex]
Algorithm Engineering (KIT) \hfill #2 -- #3\\
\hrule
\thispagestyle{empty}
}

\begin{document}
\exhead{1}{2024-04-23}{2024-05-07}

\section{Coloring Competition (Points 4)}

Implement SAT based graph vertex coloring solver.
Your application should take as a single command line argument a DIMACS file with a graph and find the smallest number of colors needed to color the graph.
The application should output the number of colors required.
The fastest solver gets \highl{seven bonus points}.
Download and test with the following benchmark instances: \url{https://github.com/satlecture/kit2024/tree/main/exercises/coloring}

\section{Sudoku Competition (Points 7)}

Write an encoder for the generalized Sudoku puzzle.
The generalized Sudoku puzzle of order $n$ is an $n^2 \times n^2$ grid, consisting of $n^2$ sub-blocks of size $n \times n$, to be filled with numbers $1, \dots, n^2$, such that
\begin{itemize}\setlength{\itemsep}{0pt}
\item in each row each number occurs exactly once,
\item in each column each number occurs exactly once, and
\item in each sub-block each number occurs exactly once.
\end{itemize}
The well-known Sudoku problem\footnote{\url{https://en.wikipedia.org/wiki/Sudoku}} is the generalized Sudoku puzzle of order three.
The best encoding (solving the most instances and fastest) will get a \highl{bonus of seven points}.
Download and test with the following benchmark instances: \url{https://github.com/satlecture/kit2024/tree/main/exercises/sudoku}.

% \section{van der Waerden Numbers (Points 3)}
% Calcucate how many variables and clauses are in the SAT encoding presented in the lecture for 
% checking whether $W(2,k) > n$?

\section{Pythagorean Triples (Points 6)}

Find a coloring for the numbers $1 \leq i \leq 1000$ such that no Pythagorean triple is monochromatic.
Estimate the number of variables and clauses in the Pythagorean triples encoding from the lecture (as a function of $n$).

\section{Tseitin Encoding (Points 6)}

Encode the following formula into CNF using the Tseitin Encoding. How would the formula look like the Plaisted-Greenbaum Encoding?
\begin{align*}
(\overline{x_1} \wedge \overline{(x_3 \iff x_2)}) \vee ((x_3 \rightarrow \overline{x_4}) \wedge (x_1 \rightarrow (x_2 \wedge \overline{x_3})) \wedge (x_4))
\end{align*}

\end{document}